% Лабораторний проєкт 1 — Алгоритми сортування
% Кіщук Ярослав Ярославович
\documentclass[12pt,a4paper]{article}
\usepackage[utf8]{inputenc}
\usepackage[T2A]{fontenc}
\usepackage[ukrainian]{babel}
\usepackage{amsmath,amssymb}
\usepackage{booktabs}
\usepackage{longtable}
\usepackage{graphicx}
\usepackage{caption}
\usepackage[margin=2.5cm]{geometry}
\usepackage{hyperref}
\graphicspath{{./}}

\title{\textbf{ОПЕРАЦІЙНІ СИСТЕМИ} \\ Лабораторний проєкт 1 \\ Алгоритми сортування}
\author{Кіщук Ярослав Ярославович}
\date{}

\begin{document}
\maketitle
\thispagestyle{empty}
\begin{center}
Міністерство освіти і науки України \\
Київський національний університет імені Тараса Шевченка \\
Факультет комп'ютерних наук та кібернетики \\
\bigskip
\textbf{ОПЕРАЦІЙНІ СИСТЕМИ} \\
Лабораторний проєкт 1 \\
\textbf{Алгоритми сортування} \\
\vfill
Виконав: студент групи \underline{\hspace{4cm}} Кіщук Ярослав Ярославович \\
Перевірила: \underline{\hspace{5cm}} \\
\vfill
Київ --- 2026
\end{center}
\newpage

\section{Мета роботи}
\label{sec:meta}
Опанувати реалізацію та експериментальне дослідження класичних і багатопотокових алгоритмів сортування масивів; порівняти їх за часом виконання, кількістю порівнянь і об’ємом додаткової пам’яті для різних розмірів даних та розподілів (найгірший / рівномірний / нормальний / найкращий випадок); перевірити відповідність вимірювань теоретичній складності та зробити висновки щодо доцільності вибору алгоритму залежно від умов.

\section{Завдання}
Реалізувати та дослідити алгоритми сортування масивів для типів \texttt{int} та \texttt{double}. Для кожного алгоритму оцінити:
\begin{itemize}
\item об’єм додаткової пам’яті $V$ (байти);
\item кількість порівнянь $K$;
\item час виконання $T$ (мс).
\end{itemize}
Теоретично оцінити складність алгоритмів. Провести експерименти для розмірів $N \in \{10^4, 2\cdot10^4, 5\cdot10^4, 10^5, 2\cdot10^5, 5\cdot10^5, 10^6\}$ та чотирьох умов даних: найгірший випадок, рівномірний розподіл, нормальний розподіл, найкращий випадок. Результати подати у вигляді таблиць; за бажанням --- графіків.

\section{Реалізація підрахунку метрик}
Метрики $V$, $K$ і $T$ збираються в одному проході експерименту так.

\textbf{Кількість операцій $K$.} Використовується тип-обгортка \texttt{Counted<T>} (\texttt{instrumented\_type.hpp}): елементи масиву мають тип \texttt{Counted<int>} або \texttt{Counted<double>}. Кожне порівняння (оператори \texttt{<}, \texttt{<=}, \texttt{>}, \texttt{>=}, \texttt{==}, \texttt{!=}) та кожне присвоєння, копіювання або переміщення інкрементує thread-local лічильники (\texttt{InstrumentationStats}). Після виклику \texttt{sort()} зчитується сума порівнянь і присвоєнь (\texttt{getTotalOperations()}); це значення записується як $K$. Таким чином $K$ відображає кількість елементарних операцій алгоритму над елементами.

\textbf{Додаткова пам’ять $V$ (байти).} Під час збірки бенчмарку підключається модуль \texttt{memory\_tracker.hpp}, який перевизначає глобальні \texttt{operator new[]} та \texttt{operator delete[]}. Кожна алокація додається до трекера (\texttt{MemoryTracker::add}), при звільненні віднімається (\texttt{subtract}). Перед кожним запуском сортування лічильник скидається; після \texttt{sort()} зчитується пікове значення \texttt{peak\_bytes} --- максимальний сумарний об’єм пам’яті, виділений одночасно під час сортування. Це і є $V$.

\textbf{Час $T$ (мс).} Вимірюється за допомогою \texttt{std::chrono::steady\_clock}: фіксується момент до виклику \texttt{strategy->sort()} і після нього; різниця переводиться в мілісекунди. Для умов ``Рівномірний'' та ``Нормальний'' виконується кілька прогонів (наприклад 100), і в таблицю записується середнє значення $T$; для ``Найгірший'' та ``Найкращий'' --- один прогон. Перед кожним прогоном генерується новий масив, скидаються трекер пам’яті та інструментація, щоб вимірювання не залежали від попередніх запусків.

\section{Теоретична складність алгоритмів}
\label{sec:complexity}
\begin{itemize}
\item \textbf{InsertionSort} --- $O(n^2)$; додаткова пам’ять $O(1)$.
\item \textbf{QuickSort} --- $O(n^2)$ у найгіршому, $O(n \log n)$ в середньому; додаткова пам’ять $O(\log n)$ (стека).
\item \textbf{MergeSort} --- $O(n \log n)$; додаткова пам’ять $O(n)$.
\item \textbf{HeapSort} --- $O(n \log n)$; додаткова пам’ять $O(1)$.
\item \textbf{BubbleSort} --- $O(n^2)$; додаткова пам’ять $O(1)$.
\item \textbf{SelectionSort} --- $O(n^2)$; додаткова пам’ять $O(1)$.
\item \textbf{ShellSort} --- залежить від послідовності кроків; типово $O(n^{1.25})$--$O(n^2)$; додаткова пам’ять $O(1)$.
\item \textbf{RadixSort} (для \texttt{int}) --- $O(n \cdot k)$, де $k$ --- кількість розрядів; додаткова пам’ять $O(n + \text{розмір алфавіту})$.
\item \textbf{CountingSort} (для \texttt{int}) --- $O(n + \text{діапазон})$; додаткова пам’ять $O(\text{діапазон})$.
\item \textbf{MultiThreadedInsertionSort} --- та сама складність $O(n^2)$, паралелізація по блоках.
\item \textbf{MultiThreadedQuickSort} --- $O(n \log n)$ в середньому при паралелізації.
\item \textbf{MultiThreadedMergeSort} --- $O(n \log n)$; додаткова пам’ять $O(n)$.
\item \textbf{MultiThreadedHeapSort} --- $O(n \log n)$; додаткова пам’ять $O(1)$.
\item \textbf{MultiThreadedBubbleSort} --- $O(n^2)$; паралелізація по проходах.
\end{itemize}

\section{Характеристики системи}
Експерименти та вимірювання виконувалися на комп’ютері з такими параметрами (за виводом бенчмарку Google Benchmark та системи):
\begin{itemize}
\item \textbf{Процесор:} 10 ядер (логичних процесорів). На macOS номінальна частота процесора може не зчитуватися бенчмарком (\texttt{hw.cpufrequency} недоступний); на вимірювання часу $T$ це не впливає.
\item \textbf{Кеш L1:} Data 64\,KiB, Instruction 128\,KiB.
\item \textbf{Кеш L2:} Unified 4096\,KiB (на кожне ядро).
\item \textbf{ОС:} macOS (платформа darwin). Конкретну версію можна вказати (наприклад, з \texttt{uname -a} або «Про цей Mac»).
\item \textbf{ОЗП:} об’єм оперативної пам’яті (наприклад, 8--16\,ГіБ) варто вказати для повноти; на результати сортування в межах проведених розмірів масивів це впливає мало.
\end{itemize}
Ці характеристики впливають на абсолютні значення часу $T$ та можуть відрізнятися на іншій конфігурації; відносне порівняння алгоритмів за складністю та залежністю від $N$ залишається коректним.

\section{Таблиці з результатами експериментів}
Експерименти проведено для типів \texttt{int} та \texttt{double}. Умови: Найгірший (Worst), Рівномірний (Uniform), Нормальний (Normal), Найкращий (Best). Для повільних алгоритмів частина вимірювань при великих $N$ пропущена через таймаут (5 хв).

\subsection{Тип \texttt{int}}
\subsection*{int, $N = 10.0K$}
\begin{center}
\small
\begin{longtable}{llrrrl}
\toprule
Алгоритм & Умова & V (Б) & K & T (мс) & BigO \\
\midrule
\endfirsthead
\multicolumn{6}{l}{(продовження)} \\
\toprule
Алгоритм & Умова & V (Б) & K & T (мс) & BigO \\
\midrule
\endhead
InsertionSort & Найгірший & 0 & 100.01M & 13.10 & $0.13 N^2$ \\
InsertionSort & Рівномірний & 0 & 49.98M & 6.68 & $0.07 N^2$ \\
InsertionSort & Нормальний & 0 & 50.07M & 6.59 & $0.07 N^2$ \\
InsertionSort & Найкращий & 0 & 30.00K & 0.0050 & $0.38 N$ \\
QuickSort & Найгірший & 0 & 640.76K & 0.1180 & $0.89 NlgN$ \\
QuickSort & Рівномірний & 0 & 387.12K & 0.3760 & $2.77 NlgN$ \\
QuickSort & Нормальний & 0 & 382.45K & 0.5930 & $45.40 N$ \\
QuickSort & Найкращий & 0 & 336.51K & 0.0530 & $0.39 NlgN$ \\
MergeSort & Найгірший & 534464 & 336.24K & 0.4950 & $50.15 N$ \\
MergeSort & Рівномірний & 534464 & 387.68K & 0.6730 & $73.32 N$ \\
MergeSort & Нормальний & 534464 & 387.69K & 0.6750 & $76.64 N$ \\
MergeSort & Найкращий & 534464 & 331.84K & 0.5130 & $49.86 N$ \\
HeapSort & Найгірший & 0 & 640.33K & 0.3880 & $3.20 NlgN$ \\
HeapSort & Рівномірний & 0 & 607.97K & 0.4460 & $3.82 NlgN$ \\
HeapSort & Нормальний & 0 & 607.81K & 0.4590 & $3.77 NlgN$ \\
HeapSort & Найкращий & 0 & 576.77K & 0.4250 & $3.29 NlgN$ \\
BubbleSort & Найгірший & 0 & 199.98M & 13.00 & $0.13 N^2$ \\
BubbleSort & Рівномірний & 0 & 125.02M & 25.50 & $0.00 N^3$ \\
BubbleSort & Нормальний & 0 & 124.99M & 25.40 & $0.00 N^3$ \\
BubbleSort & Найкращий & 0 & 49.99M & 12.90 & $0.13 N^2$ \\
SelectionSort & Найгірший & 0 & 50.02M & 86.70 & $0.67 N^2$ \\
SelectionSort & Рівномірний & 0 & 50.02M & 84.80 & $0.64 N^2$ \\
SelectionSort & Нормальний & 0 & 50.02M & 87.00 & $0.64 N^2$ \\
SelectionSort & Найкращий & 0 & 50.02M & 87.50 & $0.61 N^2$ \\
ShellSort & Найгірший & 0 & 475.15K & 0.1040 & $0.77 NlgN$ \\
ShellSort & Рівномірний & 0 & 651.99K & 0.5740 & $4.48 NlgN$ \\
ShellSort & Нормальний & 0 & 647.04K & 0.5640 & $4.46 NlgN$ \\
ShellSort & Найкращий & 0 & 360.01K & 0.0630 & $0.47 NlgN$ \\
RadixSort & Найгірший & 160000 & 170.0K & 0.0640 & $7.58 N$ \\
RadixSort & Рівномірний & 200000 & 210.0K & 0.0950 & $0.64 NlgN$ \\
RadixSort & Нормальний & 200000 & 210.0K & 0.1010 & $0.66 NlgN$ \\
RadixSort & Найкращий & 160000 & 170.0K & 0.0620 & $7.38 N$ \\
CountingSort & Найгірший & 80000 & 30.0K & 0.0190 & $1.79 N$ \\
CountingSort & Рівномірний & 119992 & 30.0K & 0.0250 & $0.21 NlgN$ \\
CountingSort & Нормальний & 85316 & 30.0K & 0.0210 & $0.24 NlgN$ \\
CountingSort & Найкращий & 80000 & 30.0K & 0.0190 & $1.84 N$ \\
MultiThreadedInsertionSort & Найгірший & 822080 & 0 & 561.00 & $4.36 N^2$ \\
MultiThreadedInsertionSort & Рівномірний & 822080 & 0 & 561.00 & $4.14 N^2$ \\
MultiThreadedQuickSort & Найгірший & 112 & 25.01K & 0.0790 & $12.55 N$ \\
MultiThreadedQuickSort & Рівномірний & 112 & 24.76K & 0.2240 & $13.32 N$ \\
MultiThreadedQuickSort & Нормальний & 112 & 24.83K & 0.2230 & $13.22 N$ \\
MultiThreadedQuickSort & Найкращий & 112 & 25.01K & 0.0480 & $5.51 N$ \\
MultiThreadedMergeSort & Найгірший & 40080 & 25.0K & 0.9730 & $85328.57 lgN$ \\
MultiThreadedMergeSort & Рівномірний & 40080 & 30.00K & 0.6300 & $37.07 N$ \\
MultiThreadedMergeSort & Нормальний & 40080 & 30.00K & 0.4080 & $28.71 N$ \\
MultiThreadedMergeSort & Найкращий & 40080 & 25.0K & 0.3010 & $75623.36 lgN$ \\
MultiThreadedHeapSort & Найгірший & 0 & 640.33K & 0.3680 & $3.05 NlgN$ \\
MultiThreadedHeapSort & Рівномірний & 0 & 607.90K & 0.4280 & $3.62 NlgN$ \\
MultiThreadedHeapSort & Нормальний & 0 & 607.81K & 0.4280 & $3.58 NlgN$ \\
MultiThreadedHeapSort & Найкращий & 0 & 576.77K & 0.4060 & $3.55 NlgN$ \\
MultiThreadedBubbleSort & Найгірший & 0 & 199.98M & 369.00 & $2.13 N^2$ \\
MultiThreadedBubbleSort & Рівномірний & 0 & 125.01M & 206.00 & $1.71 N^2$ \\
MultiThreadedBubbleSort & Нормальний & 0 & 124.99M & 145.00 & $1.84 N^2$ \\
MultiThreadedBubbleSort & Найкращий & 0 & 49.99M & 12.60 & $0.13 N^2$ \\
\bottomrule
\end{longtable}
\end{center}

\subsection*{int, $N = 20.0K$}
\begin{center}
\small
\begin{longtable}{llrrrl}
\toprule
Алгоритм & Умова & V (Б) & K & T (мс) & BigO \\
\midrule
\endfirsthead
\multicolumn{6}{l}{(продовження)} \\
\toprule
Алгоритм & Умова & V (Б) & K & T (мс) & BigO \\
\midrule
\endhead
InsertionSort & Найгірший & 0 & 400.02M & 53.00 & $0.13 N^2$ \\
InsertionSort & Рівномірний & 0 & 200.03M & 26.40 & $0.07 N^2$ \\
InsertionSort & Нормальний & 0 & 199.96M & 26.10 & $0.07 N^2$ \\
InsertionSort & Найкращий & 0 & 60.00K & 0.0080 & $0.38 N$ \\
QuickSort & Найгірший & 0 & 1.41M & 0.2570 & $0.89 NlgN$ \\
QuickSort & Рівномірний & 0 & 836.24K & 0.8020 & $2.77 NlgN$ \\
QuickSort & Нормальний & 0 & 830.40K & 1.38 & $45.40 N$ \\
QuickSort & Найкращий & 0 & 722.99K & 0.1090 & $0.39 NlgN$ \\
MergeSort & Найгірший & 1148928 & 722.48K & 1.01 & $50.15 N$ \\
MergeSort & Рівномірний & 1148928 & 835.38K & 1.39 & $73.32 N$ \\
MergeSort & Нормальний & 1148928 & 835.37K & 1.41 & $76.64 N$ \\
MergeSort & Найкращий & 1148928 & 713.68K & 1.00 & $49.86 N$ \\
HeapSort & Найгірший & 0 & 1.38M & 0.8690 & $3.20 NlgN$ \\
HeapSort & Рівномірний & 0 & 1.32M & 0.9980 & $3.82 NlgN$ \\
HeapSort & Нормальний & 0 & 1.32M & 1.01 & $3.77 NlgN$ \\
HeapSort & Найкращий & 0 & 1.26M & 0.9190 & $3.29 NlgN$ \\
BubbleSort & Найгірший & 0 & 799.96M & 52.00 & $0.13 N^2$ \\
BubbleSort & Рівномірний & 0 & 499.89M & 103.00 & $0.00 N^3$ \\
BubbleSort & Нормальний & 0 & 499.76M & 103.00 & $0.00 N^3$ \\
BubbleSort & Найкращий & 0 & 199.99M & 51.40 & $0.13 N^2$ \\
SelectionSort & Найгірший & 0 & 200.05M & 335.00 & $0.67 N^2$ \\
SelectionSort & Рівномірний & 0 & 200.05M & 328.00 & $0.64 N^2$ \\
SelectionSort & Нормальний & 0 & 200.05M & 330.00 & $0.64 N^2$ \\
SelectionSort & Найкращий & 0 & 200.05M & 330.00 & $0.61 N^2$ \\
ShellSort & Найгірший & 0 & 1.03M & 0.2220 & $0.77 NlgN$ \\
ShellSort & Рівномірний & 0 & 1.50M & 1.25 & $4.48 NlgN$ \\
ShellSort & Нормальний & 0 & 1.48M & 1.23 & $4.46 NlgN$ \\
ShellSort & Найкращий & 0 & 780.01K & 0.1360 & $0.47 NlgN$ \\
RadixSort & Найгірший & 400000 & 420.0K & 0.1770 & $7.58 N$ \\
RadixSort & Рівномірний & 400000 & 420.0K & 0.1670 & $0.64 NlgN$ \\
RadixSort & Нормальний & 400000 & 420.0K & 0.2030 & $0.66 NlgN$ \\
RadixSort & Найкращий & 400000 & 420.0K & 0.1720 & $7.38 N$ \\
CountingSort & Найгірший & 160000 & 60.0K & 0.0390 & $1.79 N$ \\
CountingSort & Рівномірний & 239988 & 60.0K & 0.0790 & $0.21 NlgN$ \\
CountingSort & Нормальний & 173955 & 60.0K & 0.0450 & $0.24 NlgN$ \\
CountingSort & Найкращий & 160000 & 60.0K & 0.0400 & $1.84 N$ \\
MultiThreadedQuickSort & Найгірший & 112 & 50.01K & 0.1500 & $12.55 N$ \\
MultiThreadedQuickSort & Рівномірний & 112 & 51.07K & 0.3610 & $13.32 N$ \\
MultiThreadedQuickSort & Нормальний & 112 & 49.28K & 0.3550 & $13.22 N$ \\
MultiThreadedQuickSort & Найкращий & 112 & 50.01K & 0.0760 & $5.51 N$ \\
MultiThreadedMergeSort & Найгірший & 80080 & 50.0K & 0.5920 & $85328.57 lgN$ \\
MultiThreadedMergeSort & Рівномірний & 80080 & 60.00K & 1.64 & $37.07 N$ \\
MultiThreadedMergeSort & Нормальний & 80080 & 60.00K & 0.5100 & $28.71 N$ \\
MultiThreadedMergeSort & Найкращий & 80080 & 50.0K & 0.3930 & $75623.36 lgN$ \\
MultiThreadedHeapSort & Найгірший & 0 & 1.38M & 0.8280 & $3.05 NlgN$ \\
MultiThreadedHeapSort & Рівномірний & 0 & 1.32M & 0.9580 & $3.62 NlgN$ \\
MultiThreadedHeapSort & Нормальний & 0 & 1.32M & 0.9550 & $3.58 NlgN$ \\
MultiThreadedHeapSort & Найкращий & 0 & 1.26M & 0.8790 & $3.55 NlgN$ \\
MultiThreadedBubbleSort & Найгірший & 0 & 799.96M & 1528 & $2.13 N^2$ \\
MultiThreadedBubbleSort & Рівномірний & 0 & 499.79M & 1008 & $1.71 N^2$ \\
\bottomrule
\end{longtable}
\end{center}

\subsection*{int, $N = 50.0K$}
\begin{center}
\small
\begin{longtable}{llrrrl}
\toprule
Алгоритм & Умова & V (Б) & K & T (мс) & BigO \\
\midrule
\endfirsthead
\multicolumn{6}{l}{(продовження)} \\
\toprule
Алгоритм & Умова & V (Б) & K & T (мс) & BigO \\
\midrule
\endhead
InsertionSort & Найгірший & 0 & 2500.05M & 357.00 & $0.13 N^2$ \\
InsertionSort & Рівномірний & 0 & 1250.54M & 164.00 & $0.07 N^2$ \\
InsertionSort & Нормальний & 0 & 1249.73M & 163.00 & $0.07 N^2$ \\
InsertionSort & Найкращий & 0 & 150.00K & 0.0200 & $0.38 N$ \\
QuickSort & Найгірший & 0 & 3.88M & 0.6890 & $0.89 NlgN$ \\
QuickSort & Рівномірний & 0 & 2.29M & 2.17 & $2.77 NlgN$ \\
QuickSort & Нормальний & 0 & 2.29M & 2.13 & $45.40 N$ \\
QuickSort & Найкращий & 0 & 2.01M & 0.3070 & $0.39 NlgN$ \\
MergeSort & Найгірший & 3137856 & 1.97M & 2.49 & $50.15 N$ \\
MergeSort & Рівномірний & 3137856 & 2.29M & 3.59 & $73.32 N$ \\
MergeSort & Нормальний & 3137856 & 2.29M & 3.85 & $76.64 N$ \\
MergeSort & Найкращий & 3137856 & 1.95M & 2.49 & $49.86 N$ \\
HeapSort & Найгірший & 0 & 3.78M & 2.43 & $3.20 NlgN$ \\
HeapSort & Рівномірний & 0 & 3.62M & 2.78 & $3.82 NlgN$ \\
HeapSort & Нормальний & 0 & 3.62M & 2.83 & $3.77 NlgN$ \\
HeapSort & Найкращий & 0 & 3.46M & 2.56 & $3.29 NlgN$ \\
BubbleSort & Найгірший & 0 & 4999.90M & 324.00 & $0.13 N^2$ \\
BubbleSort & Рівномірний & 0 & 3123.97M & 676.00 & $0.00 N^3$ \\
BubbleSort & Нормальний & 0 & 3124.39M & 685.00 & $0.00 N^3$ \\
BubbleSort & Найкращий & 0 & 1249.97M & 321.00 & $0.13 N^2$ \\
SelectionSort & Найгірший & 0 & 1250.12M & 1889 & $0.67 N^2$ \\
SelectionSort & Рівномірний & 0 & 1250.12M & 1864 & $0.64 N^2$ \\
ShellSort & Найгірший & 0 & 2.84M & 0.6000 & $0.77 NlgN$ \\
ShellSort & Рівномірний & 0 & 4.48M & 3.44 & $4.48 NlgN$ \\
ShellSort & Нормальний & 0 & 4.47M & 3.42 & $4.46 NlgN$ \\
ShellSort & Найкращий & 0 & 2.10M & 0.3680 & $0.47 NlgN$ \\
RadixSort & Найгірший & 1000000 & 1.05M & 0.3840 & $7.58 N$ \\
RadixSort & Рівномірний & 1090000 & 1.14M & 0.4670 & $0.64 NlgN$ \\
RadixSort & Нормальний & 1000000 & 1.05M & 0.4270 & $0.66 NlgN$ \\
RadixSort & Найкращий & 1000000 & 1.05M & 0.3730 & $7.38 N$ \\
CountingSort & Найгірший & 400000 & 150.0K & 0.0900 & $1.79 N$ \\
CountingSort & Рівномірний & 599993 & 150.0K & 0.1580 & $0.21 NlgN$ \\
CountingSort & Нормальний & 442439 & 150.0K & 0.1280 & $0.24 NlgN$ \\
CountingSort & Найкращий & 400000 & 150.0K & 0.0930 & $1.84 N$ \\
MultiThreadedQuickSort & Найгірший & 112 & 125.01K & 0.5780 & $12.55 N$ \\
MultiThreadedQuickSort & Рівномірний & 112 & 126.24K & 0.7160 & $13.32 N$ \\
MultiThreadedQuickSort & Нормальний & 112 & 128.24K & 0.7120 & $13.22 N$ \\
MultiThreadedQuickSort & Найкращий & 112 & 125.01K & 0.2720 & $5.51 N$ \\
MultiThreadedMergeSort & Найгірший & 200080 & 125.0K & 1.21 & $85328.57 lgN$ \\
MultiThreadedMergeSort & Рівномірний & 200080 & 150.00K & 3.04 & $37.07 N$ \\
MultiThreadedMergeSort & Нормальний & 200080 & 150.00K & 1.49 & $28.71 N$ \\
MultiThreadedMergeSort & Найкращий & 200080 & 125.0K & 1.27 & $75623.36 lgN$ \\
MultiThreadedHeapSort & Найгірший & 0 & 3.78M & 2.31 & $3.05 NlgN$ \\
MultiThreadedHeapSort & Рівномірний & 0 & 3.62M & 2.71 & $3.62 NlgN$ \\
MultiThreadedHeapSort & Нормальний & 0 & 3.62M & 2.70 & $3.58 NlgN$ \\
MultiThreadedHeapSort & Найкращий & 0 & 3.46M & 2.45 & $3.55 NlgN$ \\
\bottomrule
\end{longtable}
\end{center}

\subsection*{int, $N = 100.0K$}
\begin{center}
\small
\begin{longtable}{llrrrl}
\toprule
Алгоритм & Умова & V (Б) & K & T (мс) & BigO \\
\midrule
\endfirsthead
\multicolumn{6}{l}{(продовження)} \\
\toprule
Алгоритм & Умова & V (Б) & K & T (мс) & BigO \\
\midrule
\endhead
InsertionSort & Найгірший & 0 & 10000.10M & 1337 & $0.13 N^2$ \\
InsertionSort & Рівномірний & 0 & 5001.12M & 656.00 & $0.07 N^2$ \\
InsertionSort & Нормальний & 0 & 4999.43M & 653.00 & $0.07 N^2$ \\
InsertionSort & Найкращий & 0 & 300.00K & 0.0370 & $0.38 N$ \\
QuickSort & Найгірший & 0 & 8.42M & 1.49 & $0.89 NlgN$ \\
QuickSort & Рівномірний & 0 & 4.89M & 4.60 & $2.77 NlgN$ \\
QuickSort & Нормальний & 0 & 4.88M & 4.50 & $45.40 N$ \\
QuickSort & Найкращий & 0 & 4.27M & 0.6390 & $0.39 NlgN$ \\
MergeSort & Найгірший & 6675712 & 4.19M & 5.02 & $50.15 N$ \\
MergeSort & Рівномірний & 6675712 & 4.87M & 7.39 & $73.32 N$ \\
MergeSort & Нормальний & 6675712 & 4.87M & 7.69 & $76.64 N$ \\
MergeSort & Найкращий & 6675712 & 4.15M & 4.99 & $49.86 N$ \\
HeapSort & Найгірший & 0 & 8.07M & 5.35 & $3.20 NlgN$ \\
HeapSort & Рівномірний & 0 & 7.74M & 6.46 & $3.82 NlgN$ \\
HeapSort & Нормальний & 0 & 7.74M & 6.33 & $3.77 NlgN$ \\
HeapSort & Найкращий & 0 & 7.42M & 5.47 & $3.29 NlgN$ \\
BubbleSort & Найгірший & 0 & 19999.80M & 1292 & $0.13 N^2$ \\
BubbleSort & Рівномірний & 0 & 12502.77M & 4120 & $0.00 N^3$ \\
ShellSort & Найгірший & 0 & 6.09M & 1.28 & $0.77 NlgN$ \\
ShellSort & Рівномірний & 0 & 10.21M & 7.47 & $4.48 NlgN$ \\
ShellSort & Нормальний & 0 & 10.15M & 7.44 & $4.46 NlgN$ \\
ShellSort & Найкращий & 0 & 4.50M & 0.7780 & $0.47 NlgN$ \\
RadixSort & Найгірший & 2000000 & 2.10M & 0.7510 & $7.58 N$ \\
RadixSort & Рівномірний & 2400000 & 2.50M & 1.09 & $0.64 NlgN$ \\
RadixSort & Нормальний & 2400000 & 2.50M & 1.14 & $0.66 NlgN$ \\
RadixSort & Найкращий & 2000000 & 2.10M & 0.7320 & $7.38 N$ \\
CountingSort & Найгірший & 800000 & 300.0K & 0.1780 & $1.79 N$ \\
CountingSort & Рівномірний & 1199992 & 300.0K & 0.3540 & $0.21 NlgN$ \\
CountingSort & Нормальний & 895665 & 300.0K & 0.4300 & $0.24 NlgN$ \\
CountingSort & Найкращий & 800000 & 300.0K & 0.1830 & $1.84 N$ \\
MultiThreadedQuickSort & Найгірший & 112 & 250.01K & 1.30 & $12.55 N$ \\
MultiThreadedQuickSort & Рівномірний & 112 & 250.87K & 1.28 & $13.32 N$ \\
MultiThreadedQuickSort & Нормальний & 112 & 254.60K & 1.27 & $13.22 N$ \\
MultiThreadedQuickSort & Найкращий & 112 & 250.01K & 0.5600 & $5.51 N$ \\
MultiThreadedMergeSort & Найгірший & 400080 & 250.0K & 2.20 & $85328.57 lgN$ \\
MultiThreadedMergeSort & Рівномірний & 400080 & 300.00K & 2.91 & $37.07 N$ \\
MultiThreadedMergeSort & Нормальний & 400080 & 300.00K & 2.84 & $28.71 N$ \\
MultiThreadedMergeSort & Найкращий & 400080 & 250.0K & 2.33 & $75623.36 lgN$ \\
MultiThreadedHeapSort & Найгірший & 0 & 8.07M & 5.11 & $3.05 NlgN$ \\
MultiThreadedHeapSort & Рівномірний & 0 & 7.74M & 6.08 & $3.62 NlgN$ \\
MultiThreadedHeapSort & Нормальний & 0 & 7.74M & 6.01 & $3.58 NlgN$ \\
MultiThreadedHeapSort & Найкращий & 0 & 7.42M & 6.07 & $3.55 NlgN$ \\
\bottomrule
\end{longtable}
\end{center}

\subsection*{int, $N = 200.0K$}
\begin{center}
\small
\begin{longtable}{llrrrl}
\toprule
Алгоритм & Умова & V (Б) & K & T (мс) & BigO \\
\midrule
\endfirsthead
\multicolumn{6}{l}{(продовження)} \\
\toprule
Алгоритм & Умова & V (Б) & K & T (мс) & BigO \\
\midrule
\endhead
InsertionSort & Найгірший & 0 & 40000.20M & 5171 & $0.13 N^2$ \\
InsertionSort & Рівномірний & 0 & 19998.26M & 2596 & $0.07 N^2$ \\
QuickSort & Найгірший & 0 & 18.16M & 3.06 & $0.89 NlgN$ \\
QuickSort & Рівномірний & 0 & 10.35M & 9.36 & $2.77 NlgN$ \\
QuickSort & Нормальний & 0 & 10.32M & 142.52 & $45.40 N$ \\
QuickSort & Найкращий & 0 & 9.04M & 1.32 & $0.39 NlgN$ \\
MergeSort & Найгірший & 14151424 & 8.88M & 16.64 & $50.15 N$ \\
MergeSort & Рівномірний & 14151424 & 10.35M & 29.59 & $73.32 N$ \\
MergeSort & Нормальний & 14151424 & 10.35M & 30.12 & $76.64 N$ \\
MergeSort & Найкращий & 14151424 & 8.81M & 17.00 & $49.86 N$ \\
HeapSort & Найгірший & 0 & 17.12M & 20.87 & $3.20 NlgN$ \\
HeapSort & Рівномірний & 0 & 16.49M & 22.79 & $3.82 NlgN$ \\
HeapSort & Нормальний & 0 & 16.48M & 22.43 & $3.77 NlgN$ \\
HeapSort & Найкращий & 0 & 15.84M & 17.83 & $3.29 NlgN$ \\
ShellSort & Найгірший & 0 & 12.98M & 3.01 & $0.77 NlgN$ \\
ShellSort & Рівномірний & 0 & 23.29M & 17.83 & $4.48 NlgN$ \\
ShellSort & Нормальний & 0 & 23.15M & 19.42 & $4.46 NlgN$ \\
ShellSort & Найкращий & 0 & 9.60M & 1.94 & $0.47 NlgN$ \\
RadixSort & Найгірший & 4800000 & 5.0M & 2.15 & $7.58 N$ \\
RadixSort & Рівномірний & 4800000 & 5.0M & 2.09 & $0.64 NlgN$ \\
RadixSort & Нормальний & 4800000 & 5.0M & 2.50 & $0.66 NlgN$ \\
RadixSort & Найкращий & 4800000 & 5.0M & 2.10 & $7.38 N$ \\
CountingSort & Найгірший & 1600000 & 600.0K & 0.4734 & $1.79 N$ \\
CountingSort & Рівномірний & 2399991 & 600.0K & 2.12 & $0.21 NlgN$ \\
CountingSort & Нормальний & 1809615 & 600.0K & 1.25 & $0.24 NlgN$ \\
CountingSort & Найкращий & 1600000 & 600.0K & 0.6880 & $1.84 N$ \\
MultiThreadedQuickSort & Найгірший & 112 & 500.01K & 3.29 & $12.55 N$ \\
MultiThreadedQuickSort & Рівномірний & 112 & 485.45K & 7.72 & $13.32 N$ \\
MultiThreadedQuickSort & Нормальний & 112 & 513.16K & 3.14 & $13.22 N$ \\
MultiThreadedQuickSort & Найкращий & 112 & 500.01K & 1.42 & $5.51 N$ \\
MultiThreadedMergeSort & Найгірший & 800080 & 500.0K & 5.65 & $85328.57 lgN$ \\
MultiThreadedMergeSort & Рівномірний & 800080 & 600.00K & 7.01 & $37.07 N$ \\
MultiThreadedMergeSort & Нормальний & 800080 & 600.00K & 6.84 & $28.71 N$ \\
MultiThreadedMergeSort & Найкращий & 800080 & 500.0K & 5.44 & $75623.36 lgN$ \\
MultiThreadedHeapSort & Найгірший & 0 & 17.12M & 10.48 & $3.05 NlgN$ \\
MultiThreadedHeapSort & Рівномірний & 0 & 16.49M & 12.36 & $3.62 NlgN$ \\
MultiThreadedHeapSort & Нормальний & 0 & 16.48M & 12.28 & $3.58 NlgN$ \\
MultiThreadedHeapSort & Найкращий & 0 & 15.84M & 10.11 & $3.55 NlgN$ \\
\bottomrule
\end{longtable}
\end{center}

\subsection*{int, $N = 500.0K$}
\begin{center}
\small
\begin{longtable}{llrrrl}
\toprule
Алгоритм & Умова & V (Б) & K & T (мс) & BigO \\
\midrule
\endfirsthead
\multicolumn{6}{l}{(продовження)} \\
\toprule
Алгоритм & Умова & V (Б) & K & T (мс) & BigO \\
\midrule
\endhead
QuickSort & Найгірший & 0 & 49.91M & 7.91 & $0.89 NlgN$ \\
QuickSort & Рівномірний & 0 & 27.72M & 23.81 & $2.77 NlgN$ \\
QuickSort & Нормальний & 0 & 27.89M & 30.00 & $45.40 N$ \\
QuickSort & Найкращий & 0 & 23.67M & 5.97 & $0.39 NlgN$ \\
MergeSort & Найгірший & 37902848 & 23.73M & 41.04 & $50.15 N$ \\
MergeSort & Рівномірний & 37902848 & 27.79M & 74.85 & $73.32 N$ \\
MergeSort & Нормальний & 37902848 & 27.79M & 17967 & $76.64 N$ \\
HeapSort & Найгірший & 0 & 45.90M & 50.75 & $3.20 NlgN$ \\
HeapSort & Рівномірний & 0 & 44.47M & 65.28 & $3.82 NlgN$ \\
HeapSort & Нормальний & 0 & 44.44M & 63.45 & $3.77 NlgN$ \\
HeapSort & Найкращий & 0 & 42.98M & 43.81 & $3.29 NlgN$ \\
ShellSort & Найгірший & 0 & 33.86M & 8.53 & $0.77 NlgN$ \\
ShellSort & Рівномірний & 0 & 66.28M & 53.94 & $4.48 NlgN$ \\
ShellSort & Нормальний & 0 & 66.01M & 46.70 & $4.46 NlgN$ \\
ShellSort & Найкращий & 0 & 25.50M & 4.61 & $0.47 NlgN$ \\
RadixSort & Найгірший & 12000000 & 12.50M & 4.69 & $7.58 N$ \\
RadixSort & Рівномірний & 12680000 & 13.18M & 5.70 & $0.64 NlgN$ \\
RadixSort & Нормальний & 12000000 & 12.50M & 5.71 & $0.66 NlgN$ \\
RadixSort & Найкращий & 12000000 & 12.50M & 4.89 & $7.38 N$ \\
CountingSort & Найгірший & 4000000 & 1.50M & 1.97 & $1.79 N$ \\
CountingSort & Рівномірний & 5999990 & 1.50M & 3.62 & $0.21 NlgN$ \\
CountingSort & Нормальний & 4578449 & 1.50M & 2.70 & $0.24 NlgN$ \\
CountingSort & Найкращий & 4000000 & 1.50M & 1.22 & $1.84 N$ \\
MultiThreadedQuickSort & Найгірший & 112 & 1.25M & 7.00 & $12.55 N$ \\
MultiThreadedQuickSort & Рівномірний & 112 & 1.26M & 6.24 & $13.32 N$ \\
MultiThreadedQuickSort & Нормальний & 112 & 1.21M & 6.13 & $13.22 N$ \\
MultiThreadedQuickSort & Найкращий & 112 & 1.25M & 2.63 & $5.51 N$ \\
MultiThreadedMergeSort & Найгірший & 2000080 & 1.25M & 12.57 & $85328.57 lgN$ \\
MultiThreadedMergeSort & Рівномірний & 2000080 & 1.50M & 15.56 & $37.07 N$ \\
MultiThreadedMergeSort & Нормальний & 2000080 & 1.50M & 14.17 & $28.71 N$ \\
MultiThreadedMergeSort & Найкращий & 2000080 & 1.25M & 13.04 & $75623.36 lgN$ \\
MultiThreadedHeapSort & Найгірший & 0 & 45.90M & 27.83 & $3.05 NlgN$ \\
MultiThreadedHeapSort & Рівномірний & 0 & 44.47M & 34.93 & $3.62 NlgN$ \\
MultiThreadedHeapSort & Нормальний & 0 & 44.44M & 39.03 & $3.58 NlgN$ \\
MultiThreadedHeapSort & Найкращий & 0 & 42.98M & 27.00 & $3.55 NlgN$ \\
\bottomrule
\end{longtable}
\end{center}

\subsection*{int, $N = 1.0M$}
\begin{center}
\small
\begin{longtable}{llrrrl}
\toprule
Алгоритм & Умова & V (Б) & K & T (мс) & BigO \\
\midrule
\endfirsthead
\multicolumn{6}{l}{(продовження)} \\
\toprule
Алгоритм & Умова & V (Б) & K & T (мс) & BigO \\
\midrule
\endhead
QuickSort & Найгірший & 0 & 106.44M & 29.99 & $0.89 NlgN$ \\
QuickSort & Рівномірний & 0 & 58.63M & 67.02 & $2.77 NlgN$ \\
QuickSort & Нормальний & 0 & 59.08M & 64.48 & $45.40 N$ \\
QuickSort & Найкращий & 0 & 49.85M & 8.85 & $0.39 NlgN$ \\
HeapSort & Найгірший & 0 & 97.06M & 95.90 & $3.20 NlgN$ \\
HeapSort & Рівномірний & 0 & 93.94M & 124.60 & $3.82 NlgN$ \\
HeapSort & Нормальний & 0 & 93.88M & 130.07 & $3.77 NlgN$ \\
HeapSort & Найкращий & 0 & 91.00M & 105.11 & $3.29 NlgN$ \\
ShellSort & Найгірший & 0 & 71.72M & 16.21 & $0.77 NlgN$ \\
ShellSort & Рівномірний & 0 & 152.69M & 103.76 & $4.48 NlgN$ \\
ShellSort & Нормальний & 0 & 152.41M & 103.87 & $4.46 NlgN$ \\
ShellSort & Найкращий & 0 & 54.00M & 9.80 & $0.47 NlgN$ \\
RadixSort & Найгірший & 24000000 & 25.0M & 10.65 & $7.58 N$ \\
RadixSort & Рівномірний & 28000000 & 29.0M & 13.91 & $0.64 NlgN$ \\
RadixSort & Нормальний & 28000000 & 29.0M & 14.37 & $0.66 NlgN$ \\
CountingSort & Найгірший & 8000000 & 3.0M & 2.44 & $1.79 N$ \\
CountingSort & Рівномірний & 11999992 & 3.0M & 4.17 & $0.21 NlgN$ \\
CountingSort & Нормальний & 9226813 & 3.0M & 3.39 & $0.24 NlgN$ \\
CountingSort & Найкращий & 8000000 & 3.0M & 1.89 & $1.84 N$ \\
MultiThreadedQuickSort & Найгірший & 112 & 2.50M & 17.43 & $12.55 N$ \\
MultiThreadedQuickSort & Рівномірний & 112 & 2.51M & 12.56 & $13.32 N$ \\
MultiThreadedQuickSort & Нормальний & 112 & 2.53M & 12.17 & $13.22 N$ \\
MultiThreadedQuickSort & Найкращий & 112 & 2.50M & 5.88 & $5.51 N$ \\
MultiThreadedMergeSort & Найгірший & 4000080 & 2.50M & 26.35 & $85328.57 lgN$ \\
MultiThreadedMergeSort & Рівномірний & 4000080 & 3.00M & 30.88 & $37.07 N$ \\
MultiThreadedMergeSort & Нормальний & 4000080 & 3.00M & 29.14 & $28.71 N$ \\
MultiThreadedMergeSort & Найкращий & 4000080 & 2.50M & 21.26 & $75623.36 lgN$ \\
MultiThreadedHeapSort & Найгірший & 0 & 97.06M & 58.06 & $3.05 NlgN$ \\
MultiThreadedHeapSort & Рівномірний & 0 & 93.94M & 78.16 & $3.62 NlgN$ \\
MultiThreadedHeapSort & Нормальний & 0 & 93.88M & 2624 & $3.58 NlgN$ \\
MultiThreadedHeapSort & Найкращий & 0 & 91.00M & 57.72 & $3.55 NlgN$ \\
\bottomrule
\end{longtable}
\end{center}


\subsection{Тип \texttt{double}}
Для \texttt{double} не використовуються RadixSort та CountingSort (призначені для цілих).
\subsection*{double, $N = 10.0K$}
\begin{center}
\small
\begin{longtable}{llrrr}
\toprule
Алгоритм & Умова & V (Б) & K & T (мс) \\
\midrule
\endfirsthead
\multicolumn{5}{l}{(продовження)} \\
\toprule
Алгоритм & Умова & V (Б) & K & T (мс) \\
\midrule
\endhead
InsertionSort & Найгірший & 0 & 100.01M & 14.69 \\
InsertionSort & Рівномірний & 0 & 50.03M & 6.95 \\
InsertionSort & Нормальний & 0 & 50.02M & 6.94 \\
InsertionSort & Найкращий & 0 & 30.00K & 0.0067 \\
QuickSort & Найгірший & 0 & 640.76K & 0.1508 \\
QuickSort & Рівномірний & 0 & 391.79K & 0.4083 \\
QuickSort & Нормальний & 0 & 394.03K & 0.4133 \\
QuickSort & Найкращий & 0 & 336.51K & 0.0549 \\
MergeSort & Найгірший & 1068928 & 336.24K & 0.6030 \\
MergeSort & Рівномірний & 1068928 & 387.68K & 0.9403 \\
MergeSort & Нормальний & 1068928 & 387.69K & 0.9655 \\
MergeSort & Найкращий & 1068928 & 331.84K & 0.5715 \\
HeapSort & Найгірший & 0 & 640.33K & 0.8332 \\
HeapSort & Рівномірний & 0 & 607.96K & 0.7665 \\
HeapSort & Нормальний & 0 & 607.99K & 0.7262 \\
HeapSort & Найкращий & 0 & 576.77K & 0.6353 \\
BubbleSort & Найгірший & 0 & 199.98M & 15.96 \\
BubbleSort & Рівномірний & 0 & 124.92M & 1294 \\
BubbleSort & Нормальний & 0 & 125.02M & 28.22 \\
BubbleSort & Найкращий & 0 & 49.99M & 13.53 \\
SelectionSort & Найгірший & 0 & 50.02M & 177.53 \\
SelectionSort & Рівномірний & 0 & 50.02M & 140.36 \\
SelectionSort & Нормальний & 0 & 50.02M & 127.53 \\
SelectionSort & Найкращий & 0 & 50.02M & 127.25 \\
ShellSort & Найгірший & 0 & 475.15K & 0.1447 \\
ShellSort & Рівномірний & 0 & 658.39K & 0.6809 \\
ShellSort & Нормальний & 0 & 658.01K & 0.6797 \\
ShellSort & Найкращий & 0 & 360.01K & 0.0647 \\
MultiThreadedInsertionSort & Найгірший & 822080 & 0 & 599.38 \\
MultiThreadedInsertionSort & Рівномірний & 822080 & 0 & 23114 \\
MultiThreadedQuickSort & Найгірший & 112 & 25.01K & 0.1212 \\
MultiThreadedQuickSort & Рівномірний & 112 & 23.76K & 0.2737 \\
MultiThreadedQuickSort & Нормальний & 112 & 24.08K & 0.2731 \\
MultiThreadedQuickSort & Найкращий & 112 & 25.01K & 0.0521 \\
MultiThreadedMergeSort & Найгірший & 80080 & 25.0K & 0.6927 \\
MultiThreadedMergeSort & Рівномірний & 80080 & 30.00K & 0.9708 \\
MultiThreadedMergeSort & Нормальний & 80080 & 30.00K & 0.9728 \\
MultiThreadedMergeSort & Найкращий & 80080 & 25.0K & 0.6365 \\
MultiThreadedHeapSort & Найгірший & 0 & 640.33K & 1.33 \\
MultiThreadedHeapSort & Рівномірний & 0 & 607.95K & 1.22 \\
MultiThreadedHeapSort & Нормальний & 0 & 607.98K & 1.11 \\
MultiThreadedHeapSort & Найкращий & 0 & 576.77K & 0.9973 \\
MultiThreadedBubbleSort & Найгірший & 0 & 199.98M & 212.66 \\
MultiThreadedBubbleSort & Рівномірний & 0 & 124.95M & 5658 \\
\bottomrule
\end{longtable}
\end{center}

\subsection*{double, $N = 20.0K$}
\begin{center}
\small
\begin{longtable}{llrrr}
\toprule
Алгоритм & Умова & V (Б) & K & T (мс) \\
\midrule
\endfirsthead
\multicolumn{5}{l}{(продовження)} \\
\toprule
Алгоритм & Умова & V (Б) & K & T (мс) \\
\midrule
\endhead
InsertionSort & Найгірший & 0 & 400.02M & 52.21 \\
InsertionSort & Рівномірний & 0 & 200.11M & 33.45 \\
InsertionSort & Нормальний & 0 & 200.05M & 26.45 \\
InsertionSort & Найкращий & 0 & 60.00K & 0.0107 \\
QuickSort & Найгірший & 0 & 1.41M & 0.2803 \\
QuickSort & Рівномірний & 0 & 837.14K & 0.8909 \\
QuickSort & Нормальний & 0 & 843.85K & 0.8592 \\
QuickSort & Найкращий & 0 & 722.99K & 0.1080 \\
MergeSort & Найгірший & 2297856 & 722.48K & 1.15 \\
MergeSort & Рівномірний & 2297856 & 835.35K & 1.95 \\
MergeSort & Нормальний & 2297856 & 835.36K & 2.01 \\
MergeSort & Найкращий & 2297856 & 713.68K & 1.17 \\
HeapSort & Найгірший & 0 & 1.38M & 1.42 \\
HeapSort & Рівномірний & 0 & 1.32M & 1.60 \\
HeapSort & Нормальний & 0 & 1.32M & 1.66 \\
HeapSort & Найкращий & 0 & 1.26M & 1.43 \\
BubbleSort & Найгірший & 0 & 799.96M & 69.69 \\
BubbleSort & Рівномірний & 0 & 499.72M & 113.56 \\
BubbleSort & Нормальний & 0 & 499.87M & 115.18 \\
BubbleSort & Найкращий & 0 & 199.99M & 56.25 \\
SelectionSort & Найгірший & 0 & 200.05M & 493.90 \\
SelectionSort & Рівномірний & 0 & 200.05M & 1349 \\
SelectionSort & Нормальний & 0 & 200.05M & 29449 \\
ShellSort & Найгірший & 0 & 1.03M & 0.2329 \\
ShellSort & Рівномірний & 0 & 1.51M & 1.48 \\
ShellSort & Нормальний & 0 & 1.51M & 1.42 \\
ShellSort & Найкращий & 0 & 780.01K & 0.1455 \\
MultiThreadedQuickSort & Найгірший & 112 & 50.01K & 0.1833 \\
MultiThreadedQuickSort & Рівномірний & 112 & 48.30K & 0.4609 \\
MultiThreadedQuickSort & Нормальний & 112 & 51.12K & 0.5131 \\
MultiThreadedQuickSort & Найкращий & 112 & 50.01K & 0.1095 \\
MultiThreadedMergeSort & Найгірший & 160080 & 50.0K & 0.8341 \\
MultiThreadedMergeSort & Рівномірний & 160080 & 60.00K & 1.23 \\
MultiThreadedMergeSort & Нормальний & 160080 & 60.00K & 1.23 \\
MultiThreadedMergeSort & Найкращий & 160080 & 50.0K & 1.14 \\
MultiThreadedHeapSort & Найгірший & 0 & 1.38M & 2.19 \\
MultiThreadedHeapSort & Рівномірний & 0 & 1.32M & 2.12 \\
MultiThreadedHeapSort & Нормальний & 0 & 1.32M & 1.85 \\
MultiThreadedHeapSort & Найкращий & 0 & 1.26M & 1.50 \\
\bottomrule
\end{longtable}
\end{center}

\subsection*{double, $N = 50.0K$}
\begin{center}
\small
\begin{longtable}{llrrr}
\toprule
Алгоритм & Умова & V (Б) & K & T (мс) \\
\midrule
\endfirsthead
\multicolumn{5}{l}{(продовження)} \\
\toprule
Алгоритм & Умова & V (Б) & K & T (мс) \\
\midrule
\endhead
InsertionSort & Найгірший & 0 & 2500.05M & 331.95 \\
InsertionSort & Рівномірний & 0 & 1249.98M & 164.21 \\
InsertionSort & Нормальний & 0 & 1250.40M & 164.29 \\
InsertionSort & Найкращий & 0 & 150.00K & 0.0292 \\
QuickSort & Найгірший & 0 & 3.88M & 0.7327 \\
QuickSort & Рівномірний & 0 & 2.29M & 2.36 \\
QuickSort & Нормальний & 0 & 2.30M & 2.39 \\
QuickSort & Найкращий & 0 & 2.01M & 0.3049 \\
MergeSort & Найгірший & 6275712 & 1.97M & 2.93 \\
MergeSort & Рівномірний & 6275712 & 2.29M & 5.48 \\
MergeSort & Нормальний & 6275712 & 2.29M & 5.76 \\
MergeSort & Найкращий & 6275712 & 1.95M & 3.26 \\
HeapSort & Найгірший & 0 & 3.78M & 4.08 \\
HeapSort & Рівномірний & 0 & 3.62M & 6.18 \\
HeapSort & Нормальний & 0 & 3.62M & 7.97 \\
HeapSort & Найкращий & 0 & 3.46M & 5.89 \\
BubbleSort & Найгірший & 0 & 4999.90M & 484.32 \\
BubbleSort & Рівномірний & 0 & 3125.59M & 27005 \\
ShellSort & Найгірший & 0 & 2.84M & 0.5528 \\
ShellSort & Рівномірний & 0 & 4.51M & 3.94 \\
ShellSort & Нормальний & 0 & 4.50M & 4.03 \\
ShellSort & Найкращий & 0 & 2.10M & 0.3348 \\
MultiThreadedQuickSort & Найгірший & 112 & 125.01K & 0.6937 \\
MultiThreadedQuickSort & Рівномірний & 112 & 125.44K & 1.23 \\
MultiThreadedQuickSort & Нормальний & 112 & 121.42K & 1.19 \\
MultiThreadedQuickSort & Найкращий & 112 & 125.01K & 0.4210 \\
MultiThreadedMergeSort & Найгірший & 400080 & 125.0K & 2.73 \\
MultiThreadedMergeSort & Рівномірний & 400080 & 150.00K & 2.86 \\
MultiThreadedMergeSort & Нормальний & 400080 & 150.00K & 2.83 \\
MultiThreadedMergeSort & Найкращий & 400080 & 125.0K & 1.80 \\
MultiThreadedHeapSort & Найгірший & 0 & 3.78M & 4.30 \\
MultiThreadedHeapSort & Рівномірний & 0 & 3.62M & 4.71 \\
MultiThreadedHeapSort & Нормальний & 0 & 3.62M & 4.29 \\
MultiThreadedHeapSort & Найкращий & 0 & 3.46M & 3.42 \\
\bottomrule
\end{longtable}
\end{center}

\subsection*{double, $N = 100.0K$}
\begin{center}
\small
\begin{longtable}{llrrr}
\toprule
Алгоритм & Умова & V (Б) & K & T (мс) \\
\midrule
\endfirsthead
\multicolumn{5}{l}{(продовження)} \\
\toprule
Алгоритм & Умова & V (Б) & K & T (мс) \\
\midrule
\endhead
InsertionSort & Найгірший & 0 & 10000.10M & 1321 \\
InsertionSort & Рівномірний & 0 & 4997.28M & 12610 \\
QuickSort & Найгірший & 0 & 8.42M & 1.57 \\
QuickSort & Рівномірний & 0 & 4.91M & 4.93 \\
QuickSort & Нормальний & 0 & 4.89M & 4.94 \\
QuickSort & Найкращий & 0 & 4.27M & 0.6027 \\
MergeSort & Найгірший & 13351424 & 4.19M & 6.31 \\
MergeSort & Рівномірний & 13351424 & 4.87M & 12.10 \\
MergeSort & Нормальний & 13351424 & 4.87M & 12.15 \\
MergeSort & Найкращий & 13351424 & 4.15M & 6.68 \\
HeapSort & Найгірший & 0 & 8.07M & 12.92 \\
HeapSort & Рівномірний & 0 & 7.74M & 12.68 \\
HeapSort & Нормальний & 0 & 7.74M & 10.35 \\
HeapSort & Найкращий & 0 & 7.42M & 8.24 \\
ShellSort & Найгірший & 0 & 6.09M & 1.22 \\
ShellSort & Рівномірний & 0 & 10.23M & 8.26 \\
ShellSort & Нормальний & 0 & 10.22M & 8.17 \\
ShellSort & Найкращий & 0 & 4.50M & 0.6910 \\
MultiThreadedQuickSort & Найгірший & 112 & 250.01K & 2.08 \\
MultiThreadedQuickSort & Рівномірний & 112 & 255.71K & 2.25 \\
MultiThreadedQuickSort & Нормальний & 112 & 249.50K & 2.57 \\
MultiThreadedQuickSort & Найкращий & 112 & 250.01K & 1.18 \\
MultiThreadedMergeSort & Найгірший & 800080 & 250.0K & 3.45 \\
MultiThreadedMergeSort & Рівномірний & 800080 & 300.00K & 5.13 \\
MultiThreadedMergeSort & Нормальний & 800080 & 300.00K & 5.46 \\
MultiThreadedMergeSort & Найкращий & 800080 & 250.0K & 3.88 \\
MultiThreadedHeapSort & Найгірший & 0 & 8.07M & 7.54 \\
MultiThreadedHeapSort & Рівномірний & 0 & 7.74M & 8.44 \\
MultiThreadedHeapSort & Нормальний & 0 & 7.74M & 8.43 \\
MultiThreadedHeapSort & Найкращий & 0 & 7.42M & 6.72 \\
\bottomrule
\end{longtable}
\end{center}

\subsection*{double, $N = 200.0K$}
\begin{center}
\small
\begin{longtable}{llrrr}
\toprule
Алгоритм & Умова & V (Б) & K & T (мс) \\
\midrule
\endfirsthead
\multicolumn{5}{l}{(продовження)} \\
\toprule
Алгоритм & Умова & V (Б) & K & T (мс) \\
\midrule
\endhead
QuickSort & Найгірший & 0 & 18.16M & 3.37 \\
QuickSort & Рівномірний & 0 & 10.39M & 10.97 \\
QuickSort & Нормальний & 0 & 10.38M & 11.49 \\
QuickSort & Найкращий & 0 & 9.04M & 1.39 \\
MergeSort & Найгірший & 28302848 & 8.88M & 13.61 \\
MergeSort & Рівномірний & 28302848 & 10.35M & 25.87 \\
MergeSort & Нормальний & 28302848 & 10.35M & 25.37 \\
MergeSort & Найкращий & 28302848 & 8.81M & 13.74 \\
HeapSort & Найгірший & 0 & 17.12M & 18.30 \\
HeapSort & Рівномірний & 0 & 16.49M & 19.95 \\
HeapSort & Нормальний & 0 & 16.49M & 20.38 \\
HeapSort & Найкращий & 0 & 15.84M & 16.54 \\
ShellSort & Найгірший & 0 & 12.98M & 2.48 \\
ShellSort & Рівномірний & 0 & 23.33M & 19.46 \\
ShellSort & Нормальний & 0 & 23.31M & 24.31 \\
ShellSort & Найкращий & 0 & 9.60M & 1.95 \\
MultiThreadedQuickSort & Найгірший & 112 & 500.01K & 6.12 \\
MultiThreadedQuickSort & Рівномірний & 112 & 511.36K & 5.59 \\
MultiThreadedQuickSort & Нормальний & 112 & 474.76K & 4.99 \\
MultiThreadedQuickSort & Найкращий & 112 & 500.01K & 1.98 \\
MultiThreadedMergeSort & Найгірший & 1600080 & 500.0K & 8.47 \\
MultiThreadedMergeSort & Рівномірний & 1600080 & 600.00K & 10.73 \\
MultiThreadedMergeSort & Нормальний & 1600080 & 600.00K & 10.18 \\
MultiThreadedMergeSort & Найкращий & 1600080 & 500.0K & 6.59 \\
MultiThreadedHeapSort & Найгірший & 0 & 17.12M & 15.39 \\
MultiThreadedHeapSort & Рівномірний & 0 & 16.49M & 19.00 \\
MultiThreadedHeapSort & Нормальний & 0 & 16.49M & 925.38 \\
MultiThreadedHeapSort & Найкращий & 0 & 15.84M & 49.12 \\
\bottomrule
\end{longtable}
\end{center}

\subsection*{double, $N = 500.0K$}
\begin{center}
\small
\begin{longtable}{llrrr}
\toprule
Алгоритм & Умова & V (Б) & K & T (мс) \\
\midrule
\endfirsthead
\multicolumn{5}{l}{(продовження)} \\
\toprule
Алгоритм & Умова & V (Б) & K & T (мс) \\
\midrule
\endhead
QuickSort & Найгірший & 0 & 49.91M & 10.08 \\
QuickSort & Рівномірний & 0 & 27.92M & 30.86 \\
QuickSort & Нормальний & 0 & 28.03M & 32.14 \\
QuickSort & Найкращий & 0 & 23.67M & 4.15 \\
MergeSort & Найгірший & 75805696 & 23.73M & 32.92 \\
MergeSort & Рівномірний & 75805696 & 27.79M & 71.58 \\
MergeSort & Нормальний & 75805696 & 27.79M & 19096 \\
HeapSort & Найгірший & 0 & 45.90M & 47.77 \\
HeapSort & Рівномірний & 0 & 44.47M & 55.86 \\
HeapSort & Нормальний & 0 & 44.47M & 54.87 \\
HeapSort & Найкращий & 0 & 42.98M & 43.21 \\
ShellSort & Найгірший & 0 & 33.86M & 8.56 \\
ShellSort & Рівномірний & 0 & 66.38M & 67.37 \\
ShellSort & Нормальний & 0 & 66.26M & 60.82 \\
ShellSort & Найкращий & 0 & 25.50M & 4.76 \\
MultiThreadedQuickSort & Найгірший & 112 & 1.25M & 9.52 \\
MultiThreadedQuickSort & Рівномірний & 112 & 1.28M & 10.60 \\
MultiThreadedQuickSort & Нормальний & 112 & 1.21M & 9.73 \\
MultiThreadedQuickSort & Найкращий & 112 & 1.25M & 3.96 \\
MultiThreadedMergeSort & Найгірший & 4000080 & 1.25M & 17.29 \\
MultiThreadedMergeSort & Рівномірний & 4000080 & 1.50M & 25.79 \\
MultiThreadedMergeSort & Нормальний & 4000080 & 1.50M & 23.10 \\
MultiThreadedMergeSort & Найкращий & 4000080 & 1.25M & 15.21 \\
MultiThreadedHeapSort & Найгірший & 0 & 45.90M & 54.83 \\
MultiThreadedHeapSort & Рівномірний & 0 & 44.47M & 67.38 \\
MultiThreadedHeapSort & Нормальний & 0 & 44.47M & 51.99 \\
MultiThreadedHeapSort & Найкращий & 0 & 42.98M & 41.11 \\
\bottomrule
\end{longtable}
\end{center}

\subsection*{double, $N = 1.0M$}
\begin{center}
\small
\begin{longtable}{llrrr}
\toprule
Алгоритм & Умова & V (Б) & K & T (мс) \\
\midrule
\endfirsthead
\multicolumn{5}{l}{(продовження)} \\
\toprule
Алгоритм & Умова & V (Б) & K & T (мс) \\
\midrule
\endhead
QuickSort & Найгірший & 0 & 106.44M & 23.45 \\
QuickSort & Рівномірний & 0 & 58.78M & 66.09 \\
QuickSort & Нормальний & 0 & 58.77M & 63.13 \\
QuickSort & Найкращий & 0 & 49.85M & 7.28 \\
HeapSort & Найгірший & 0 & 97.06M & 91.62 \\
HeapSort & Рівномірний & 0 & 93.94M & 119.19 \\
HeapSort & Нормальний & 0 & 93.94M & 120.24 \\
HeapSort & Найкращий & 0 & 91.00M & 93.24 \\
ShellSort & Найгірший & 0 & 71.72M & 16.89 \\
ShellSort & Рівномірний & 0 & 153.39M & 128.46 \\
ShellSort & Нормальний & 0 & 153.29M & 128.81 \\
ShellSort & Найкращий & 0 & 54.00M & 10.13 \\
MultiThreadedQuickSort & Найгірший & 112 & 2.50M & 19.89 \\
MultiThreadedQuickSort & Рівномірний & 112 & 2.40M & 21.11 \\
MultiThreadedQuickSort & Нормальний & 112 & 2.52M & 20.45 \\
MultiThreadedQuickSort & Найкращий & 112 & 2.50M & 8.34 \\
MultiThreadedMergeSort & Найгірший & 8000080 & 2.50M & 30.00 \\
MultiThreadedMergeSort & Рівномірний & 8000080 & 3.00M & 49.74 \\
MultiThreadedMergeSort & Нормальний & 8000080 & 3.00M & 46.46 \\
MultiThreadedMergeSort & Найкращий & 8000080 & 2.50M & 28.03 \\
MultiThreadedHeapSort & Найгірший & 0 & 97.06M & 89.17 \\
MultiThreadedHeapSort & Рівномірний & 0 & 93.94M & 115.59 \\
MultiThreadedHeapSort & Нормальний & 0 & 93.94M & 114.05 \\
MultiThreadedHeapSort & Найкращий & 0 & 91.00M & 89.72 \\
\bottomrule
\end{longtable}
\end{center}


\section{Висновок}
\label{sec:visnovok}
\begin{itemize}
\item \textbf{Узгодження з теорією.}
InsertionSort: у найкращому випадку BigO $\approx 0.38 N$ (відповідає $O(n)$), у найгіршому/рівномірному/нормальному $\approx 0.07$--$0.13\,N^2$ ($O(n^2)$). SelectionSort: $\approx 0.61$--$0.67\,N^2$ у всіх умовах --- відповідає $O(n^2)$. HeapSort: BigO $3.2$--$3.8\,N\lg N$ --- узгоджується з $O(n\log n)$. ShellSort: $0.77$--$4.48\,N\lg N$. QuickSort у найгіршому, рівномірному та найкращому: $0.89\,N\lg N$, $2.77\,N\lg N$, $0.39\,N\lg N$. CountingSort і RadixSort: лінійні або близькі до $N\lg N$. MultiThreadedHeapSort і MultiThreadedBubbleSort (Best): BigO узгоджується з теорією.

\item \textbf{Невідповідності теорії (артефакти апроксимації).}
MergeSort --- бенчмарк підібрав модель лінійну за $N$ ($50.15 N$, $73.32 N$ тощо) замість $O(n\log n)$; \emph{не відповідає теорії} (обмежений діапазон $N$). QuickSort (Normal) --- BigO $45.40 N$ замість $O(n\log n)$; \emph{помилковий підбор моделі}. BubbleSort (Uniform, Normal) --- BigO $0.00 N^3$; теорія $O(n^2)$ --- \emph{артефакт}. MultiThreadedMergeSort --- дуже великі коефіцієнти при $\lg N$ ($85328\,\lg N$ тощо); \emph{артефакт апроксимації}. MultiThreadedQuickSort --- real\_time як $12.55 N$; теорія $O(n\log n)$; \emph{у діапазоні 10K--100K може виглядати лінійно через паралелізм}.

\item Найшвидшими на великих об’ємах залишаються CountingSort, RadixSort (для \texttt{int}), QuickSort, MergeSort, HeapSort та їхні багатопотокові варіанти; алгоритми $O(n^2)$ та їхні паралельні версії на великих $N$ значно повільніші, з таймаутами при $N \geq 10^5$--$2\cdot10^5$.

\item Багатопотокові QuickSort, MergeSort, HeapSort дають помітне прискорення за real\_time; паралельні InsertionSort і BubbleSort часто гірші за однопотокові.

\item Додаткова пам’ять $V$ узгоджується з оцінками: MergeSort/MultiThreadedMergeSort --- значне $V$, RadixSort/CountingSort --- помірне, QuickSort/HeapSort --- незначне або нульове.

\item Для типів \texttt{int} і \texttt{double} результати подібні; графіки часу від $N$ підтверджують очікувані залежності за складністю.
\end{itemize}

\section{Код програми та скріни}
Реалізація, бенчмарки та додаток для графіків доступні в репозиторії:
\begin{center}
\url{https://github.com/kshchuk/Sorting-Algorithms}
\end{center}

\subsection*{Скріншоти консолі та GUI}
Консольний вивід бенчмарку (стратегії, прогрес, таблиці CSV):
\begin{center}
\includegraphics[width=0.95\textwidth]{console.png}
\end{center}
\vspace{1em}
Графічний додаток \texttt{sorting\_charts} (вибір типу даних, запуск бенчмарків, графіки часу від $N$):
\begin{center}
\includegraphics[width=0.95\textwidth]{gui.png}
\end{center}

\subsection*{Графіки залежності часу від розміру масиву}
Графіки побудовано за результатами експериментів для розподілів Uniform та Normal (типи \texttt{int} та \texttt{double}).
\begin{center}
\includegraphics[width=0.9\textwidth]{uni-int.png}
\end{center}
\captionof{figure}{Рівномірний розподіл, тип \texttt{int}.}
\vspace{0.5em}
\begin{center}
\includegraphics[width=0.9\textwidth]{normal-int.png}
\end{center}
\captionof{figure}{Нормальний розподіл, тип \texttt{int}.}
\vspace{0.5em}
\begin{center}
\includegraphics[width=0.9\textwidth]{uni-double.png}
\end{center}
\captionof{figure}{Рівномірний розподіл, тип \texttt{double}.}
\vspace{0.5em}
\begin{center}
\includegraphics[width=0.9\textwidth]{uni-normal.png}
\end{center}
\captionof{figure}{Рівномірний та нормальний розподіли (порівняння).}
\end{document}
